\documentclass[12pt]{article}
\usepackage{amsmath}
\usepackage{amsthm}
\usepackage{amsfonts}
\usepackage{amssymb}
\usepackage{authblk}
\usepackage{tkz-euclide}
\usepackage{tikz}
\usepackage{changepage}
\usepackage{lipsum}
\usepackage{tree-dvips}
\usepackage{qtree}
\usepackage[linguistics]{forest}
\usepackage[hidelinks]{hyperref}
\usepackage{mathtools}
\usepackage{blindtext}
\usepackage[cal=esstix,frak=euler,scr=boondox,bb= pazo]{mathalfa}
\usepackage{graphicx}
\graphicspath{{./images/}}
\allowdisplaybreaks
\allowbreak
\theoremstyle{definition}
\newtheorem{definition}{Definition}
\newtheoremstyle{named}{}{}{\itshape}{}{\bfseries}{.}{.5em}{\thmnote{#3's }#1}
\theoremstyle{named}
\newtheorem*{namedconjecture}{Distinct Factorizations Conjecture}
\newtheorem{conjecture}{Conjecture}
\DeclareMathOperator{\sech}{sech}
\DeclareMathOperator{\arcsec}{arcsec}
\newcounter{customDef}
\renewcommand{\thecustomDef}{\arabic{customDef}}
\begin{document}
\title{Math 104 - Midterm 1 Definitions}
\author{}
\date{}
\maketitle
\date

\setcounter{customDef}{0}
\renewcommand{\thedefinition}{1.5}
\begin{definition}
    Let $S$ be a set. An \textbf{order} on $S$ is a relation that satisfies both of the following properties:
    \begin{enumerate} 
        \item Let $x,y \in S$. Only one of the following statements is true. $x<y, x=y, x>y$. 
        \item If $x,y,z \in S$ and $x<y$ and $y<z$, then $x<z$ and it follows that $x<y<z$.
    \end{enumerate}
\end{definition}

\setcounter{customDef}{0}
\renewcommand{\thedefinition}{1.6}
\begin{definition}
    An \textbf{ordered set} $S$ is a set in which an order is defined.
\end{definition}

\setcounter{customDef}{0}
\renewcommand{\thedefinition}{1.7}
\begin{definition}
    Suppose $S$ is an ordered set and $E \subset S$. $E$ is \textbf{bounded above} if there exists a $\beta \in S$ such that $\beta \geq x$ for all $x \in E$. $E$ is \textbf{bounded below} if there exists a $\beta \in S$ such that $\beta \leq x$ for all $x \in E$. 
\end{definition}

\setcounter{customDef}{0}
\renewcommand{\thedefinition}{1.8}
\begin{definition}
    Suppose $S$ is an ordered set, $E \subset S$, $E$ is bounded above. Suppose there exists an $\alpha \in S$ with the following two properties:
    \begin{enumerate}
        \item $\alpha$ is an upper bound of $E$.
        \item If $\gamma < \alpha$, then $\gamma$ is not an upper bound.
    \end{enumerate}
    then $\alpha$ is the \textbf{least upper-bound} of $E$, denoted by $\alpha = \sup E$. 
    \\
    Suppose there is a $\beta \in S$ with the following two properties: 
    \begin{enumerate}
        \item $\beta$ is a lower bound of $E$.
        \item If $\gamma > \beta$, then $\gamma$ is not a lower bound of $E$.
    \end{enumerate}
    then $\beta$ is the \textbf{greatest lower-bound} of $E$, denoted by $\beta = \inf E$. 
\end{definition}

\setcounter{customDef}{0}
\renewcommand{\thedefinition}{1.10}
\begin{definition}
    An ordered set $S$ is said to have the \textbf{least upper-bound property} if the following statement is true: $E \subset S$, $E$ is nonempty, $E$ is bounded above, and $\sup E \in S$. 
\end{definition}

\setcounter{customDef}{0}
\renewcommand{\thedefinition}{1.17}
\begin{definition}
    An \textbf{ordered field} is a field $F$ which is also an \textbf{ordered set} which satisfies the following two properties:
    \begin{enumerate}
        \item $x+y<x+z$ if $x,y,z \in F$ and $y<z$.
        \item $xy>0$ if $x,y \in F$, $x>0$, and $y>0$. 
    \end{enumerate}
\end{definition}

\setcounter{customDef}{0}
\renewcommand{\thedefinition}{2.1}
\begin{definition}
    Consider two sets $A$ and $B$ which can contain any objects whatsoever. Suppose that with each element $x \in A$, we associate an element in $B$ through some manner. Let this assignment be denoted by $f(x)$ where $f$ is a \textbf{function}. We can also say that there is a mapping of $A$ into $B$. We use the following notation: $f: A \to B$. $A$ is called the \textbf{domain} of $f$ (we also say that $f$ is defined on $A$). The elements $f(x)$ in $B$ are called the \textbf{values} of $f$. The set of all $f(x)$ is called the \textbf{range} of $f$. 
\end{definition}

\setcounter{customDef}{0}
\renewcommand{\thedefinition}{2.2}
\begin{definition}
    Let $f: A \to B$. If $E \subset A$, then the \textbf{image} of $E$ under $f$ is the set $\{f(x) \mid x \in E\}$. If $E \subset B$, then the \textbf{inverse image} of $E$ under $f$ is the set $\{x \in A \mid f(x) \in E\}$. If $y \in B$, $f^{-1}(y) = \{x \in A \mid f(x)=y\}$. If $f$ is \textbf{onto}, then every in element in $B$ appears in the image of $E$ under $f$. If $(f(x) = f(y))\implies(x=y)$, then $f$ is 1-1. If $f$ is both onto and 1-1, then $f$ is \textbf{bijective}. 
\end{definition}

\setcounter{customDef}{0}
\renewcommand{\thedefinition}{2.3}
\begin{definition}
    If there is a 1-1 mapping of $A$ onto $B$, then we say that $A$ and $B$ can be put into \textbf{1-1 correspondence}. If this is true, $A$ and $B$ have the same \textbf{cardinal number}, or that $A$ and $B$ are \textbf{equivalent}, denoted by $A \sim B$. If this is true, then the relation $A \sim B$ has the following 3 properties: 
    \begin{enumerate}
        \item \textbf{Reflexive}: $A \sim A$.
        \item \textbf{Symmetric}: $(A \sim B) \implies (B \sim A)$.
        \item \textbf{Transitive}: $(A \sim B \land B \sim C) \implies (A \sim C)$. 
    \end{enumerate}
    Any relation with these 3 properties is called an \textbf{equivalence relation}. 
\end{definition}

\setcounter{customDef}{0}
\renewcommand{\thedefinition}{2.4}
\begin{definition}
    Let $A$ be a set, $n \in \mathbb{N}$, $J_n$ denotes the set of the first $n$ positive integers, and $J = \mathbb{N}$. We have some terms to define: 
    \begin{enumerate}
        \item $A$ is \textbf{finite} if the relation $A \sim J_n$ exists for some $n$.
        \item $A$ is \textbf{infinite} if $A$ is not finite. 
        \item $A$ is \textbf{countable} if the relation $A \sim J$ exists. 
        \item $A$ is \textbf{uncountable} if it is not finite and not countable. 
        \item $A$ is \textbf{at most countable} if it is finite or countable. 
    \end{enumerate}
\end{definition}

\setcounter{customDef}{0}
\renewcommand{\thedefinition}{2.7}
\begin{definition}
    A sequence is a function $f(n)$ that is defined on $\mathbb{N}$. If $f(n) = x_n$ for all $n$, then we denote $\{x_n\}$ to be the entire sequence $f(n)$ applied to all $n \in \mathbb{N}$. 
\end{definition}

\setcounter{customDef}{0}
\renewcommand{\thedefinition}{2.15}
\begin{definition}
    A set $X$ is said to be a \textbf{metric space} if for every $p,q \in X$ (elements in $X$ are called \textbf{points}) there is associated a real number $d(p,q)$ that satisfies the following 3 properties (a function that has these 3 properties is also called a \textbf{distance function} or a \textbf{metric}):
    \begin{enumerate}
        \item $d(p,q) > 0$ if $p \neq q$ and $d(p,p) = 0$.
        \item $d(p,q) = d(q,p)$.
        \item $d(p,q) \leq d(p,r) + d(r,q)$ for any $r \in X$ \hspace{0.1cm} (triangle inequality). 
    \end{enumerate} 
\end{definition}

\setcounter{customDef}{0}
\renewcommand{\thedefinition}{of Discrete Metric Space}
\begin{definition}
    For any set $X$, we can define $d_D(x,y) = 0$ if $x=y$ and $d_D(x,y)=1$ if $x \neq y$. Therefore, the pair $(X,d_D)$ denote the \textbf{discrete metric space}. Specifically to when $X=\mathbb{R}^n$, we have the notation $(\mathbb{R}^n, d_D)$.
\end{definition}

\setcounter{customDef}{0}
\renewcommand{\thedefinition}{of Sequence Spaces}
\begin{definition}
    A \textbf{sequence space} is a space of all sequences of real numbers that are bounded. 
\end{definition}

\setcounter{customDef}{0}
\renewcommand{\thedefinition}{of $l^p$}
\begin{definition}
    Let $l^p$ denote the set of all sequences $\{x_i\}\biggr\rvert_{i=1}^{n}$ such that $\sum_{j=1}^{\infty} \left|x_j\right|^p < \infty$.
\end{definition}

\setcounter{customDef}{0}
\renewcommand{\thedefinition}{of $L^p$-metric}
\begin{definition}
    Under the $L^p$-metric, the standard distance function is defined as $d(x,y) = \left(\sum_{j=1}^{n} \left|x_j-y_j\right|^p \right)^\frac{1}{p}$ where $n$ represents the dimension (the same $n$ as in $\mathbb{R}^n$). For sequences, $d\left(\{x_i\}\biggr\rvert_{i=1}^{\infty}, \{y_i\}\biggr\rvert_{i=1}^{\infty}\right) = \left(\sum_{j=1}^{n} \left|x_j-y_j\right|^p \right)^\frac{1}{p}$
\end{definition}

\setcounter{customDef}{0}
\renewcommand{\thedefinition}{of Open/Closed Ball}
\begin{definition}
    Let $x \in \mathbb{R}^n$ and $r$ be a real number with $r>0$. The \textbf{open ball} with center $x$ is defined to be the set $\{y \in \mathbb{R}^n \mid d(x,y) < r\}$ and the \textbf{closed ball} with center $x$ is defined to be the set $\{y \in \mathbb{R}^n \mid d(x,y) \leq r\}$.
\end{definition}

\setcounter{customDef}{0}
\renewcommand{\thedefinition}{2.17}
\begin{definition}
    The \textbf{segment} $(a,b)$ is defined to be the set $\{x \in \mathbb{R} \mid a < x < b\}$ and the \textbf{interval} $[a,b]$ is defined to be the set $\{x \in \mathbb{R} \mid a \leq x \leq b\}$.
\end{definition}

\setcounter{customDef}{0}
\renewcommand{\thedefinition}{2.18}
\begin{definition}
    Let $(X,d)$ be a metric space. We define the following terms: 
    \begin{enumerate}
        \item Let $p$ be a point in $X$ with $p \in X$. A \textbf{neighborhood} of point $p$ is the set $N_r(p)$ with \textbf{radius} $r>0$ such that $N_r(p) = \{q \in X \mid d(p,q) < r\}$. 
        \item A point $p$ is a \textbf{limit point} of the set $E \subseteq X$ if every neighborhood of $p$ contains a point $q \neq p$ such that $q \in E$. 
        \item A point $p$ is an \textbf{isolated point} in $E$ if $p$ is not a limit point. 
        \item $E$ is \textbf{closed} if every limit point of $E$ is a point of $E$. 
        \item A point $p$ is an \textbf{interior point} of $E$ if a neighborhood $N$ of $p$ satisfies $N \subset E$. 
        \item $E$ is \textbf{open} if every point of $E$ is an interior point of $E$. 
        \item The \textbf{complement} of $E$ is the set $E^c = \{x \in X \mid x \notin E\}$. 
        \item $E$ is \textbf{perfect} if $E$ is closed and if every point of $E$ is a limit point of $E$. 
    \end{enumerate}
\end{definition}

\setcounter{customDef}{0}
\renewcommand{\thedefinition}{2.26}
\begin{definition}
    If $X$ is a metric space, $E \subset X$, and $E'$ denotes the set of all limit points of $E$, then $\bar{E} = E \cup E'$ is defined to be the \textbf{closure} of $E$.
\end{definition}

\setcounter{customDef}{0}
\renewcommand{\thedefinition}{2.31}
\begin{definition}
    An \textbf{open cover} of a set $E \subset X$ is the collection $\{G_\alpha\}$ of open subsets of $X$ such that $E \subset \cup_{\alpha}^{} G_\alpha$. A \textbf{subcover} of $E$ is a subcollection of that still contains $E$. 
\end{definition}

\setcounter{customDef}{0}
\renewcommand{\thedefinition}{2.32}
\begin{definition}
    A set $E \subset X$ is said to be \textbf{compact} if every open cover of $E$ contains a finite subcover. 
\end{definition}

\setcounter{customDef}{0}
\renewcommand{\thedefinition}{of $k$-cell}
\begin{definition}
    If $a_i < b_i$ for $i = 1,\dots,k$, the set of all points $x = \{x_1,\dots,x_k\}$ in $\mathbb{R}^k$ that satisfy $a_i \leq x_i \leq b_i$ is called a \textbf{$k$-cell}. 
\end{definition}

\end{document}