\documentclass[12pt]{article}
\usepackage{amsmath}
\usepackage{amsthm}
\usepackage{amsfonts}
\usepackage{amssymb}
\usepackage{authblk}
\usepackage{tkz-euclide}
\usepackage{tikz}
\usepackage{changepage}
\usepackage{lipsum}
\usepackage{tree-dvips}
\usepackage{qtree}
\usepackage[linguistics]{forest}
\usepackage[hidelinks]{hyperref}
\usepackage{mathtools}
\usepackage{blindtext}
\usepackage[cal=esstix,frak=euler,scr=boondox,bb= pazo]{mathalfa}
\usepackage{graphicx}
\graphicspath{{./images/}}
\allowdisplaybreaks
\allowbreak
\theoremstyle{definition}
\newtheorem{definition}{Definition}
\newtheoremstyle{named}{}{}{\itshape}{}{\bfseries}{.}{.5em}{\thmnote{#3}#1}
\theoremstyle{named}
\newtheorem*{namedconjecture}{Distinct Factorizations Conjecture}
\newtheorem{conjecture}{Conjecture}
\DeclareMathOperator{\sech}{sech}
\DeclareMathOperator{\arcsec}{arcsec}
\newcounter{customDef}
\renewcommand{\thecustomDef}{\arabic{customDef}}
\newcounter{customThm}
\renewcommand{\thecustomThm}{\arabic{customThm}}
\newtheorem{theorem}{}
\begin{document}
\title{Math 104 Lecture Notes}
\author{Chapter 4 - Continuity}
\date{July 10, 2024}
\maketitle
\date

\section*{Notes}

In previous lectures, we discussed the $\epsilon-\mathbb{N}$ definition for convergence of a sequence $\{x_n\}$, namely $\lim_{x\to\infty} x_n = a$ for some $a \in \mathbb{R}$. We would like the general approach for finding limits of the form $\lim_{x \to p} f(x) = q \hspace{0.1cm} (= f(p))$ for some finite values $p$ and $q$. First, we define a function $f: X \to Y$, where $(X, d_x)$ and $(Y, d_y)$ are metric spaces. We consider a mapping of $X$ into $Y$; in particular, we observe the behavior of $f$ as $x$ tends to $p$. 

\setcounter{customDef}{0}
\renewcommand{\thedefinition}{4.1}
\begin{definition}
    Let $(X, d_x)$ and $(Y, d_y)$ be metric spaces, $E \subset X$, $f: X \to Y$, and $p$ is a limit point of $E$. The statement ``$f(x) \to q$ as $x \to p$" is written as $\lim_{x \to p} f(x) = q$. We say that $\lim_{x \to p} f(x) = q$ if there is a point $q \in Y$ such that for every $\epsilon > 0$ there exists a $\delta > 0$ such that for all $x \in E$, if $0 < d_x(x,p) < \delta$, then $d_y(f(x),q) < \epsilon$. 
\end{definition}

Note that a function may not be defined at a certain point but the limit of the function may exist at that point. Consider the following example: 
$$
f(x) = \frac{x(x+1)}{x+1}, \hspace{0.5cm} f(-1) \hspace{0.25cm} \mathrm{DNE}, \hspace{0.5cm} \lim_{x \to -1} f(x) = -1
$$

If we take the inequality $0 < d_x(x,p) < \delta$, we don't look at what $f$ does at $p$, just what $f$ does near $p$. 



\begin{theorem}[\textbf{Theorem 4.2}]
    Let $X, Y, E, f, p$ be as from Definition 4.1. $\lim_{x \to p} f(x) = q$ if and only if for every sequence $\{p_n\}$ in $E$ such that $p_n \neq p$ and $\lim_{n \to \infty} p_n = p$, we have that $\lim_{n \to \infty} f(p_n) = q$. 
\end{theorem}
\begin{proof}
    $(\rightarrow)$: Suppose $\lim_{x \to p} f(x) = q$. Let $\{p_n\}$ in $E$ satisfy $p_n \neq p$ and $p_n \to p$. Let $\epsilon > 0$ be given. We know there exists a $\delta > 0$ such that if $x \in E$ and $0 < d_x(x,p) < \delta$, then $d_y(f(x),q) < \epsilon$. Since $p_n \to p$, we know there exists an $N \in \mathbb{N}$ such that if $n \geq N$, then $d(p_n, p) < \delta$. So for this $N$, if $n \geq N$, then $d(f(p_n), q) < \epsilon$. \\
    $(\leftarrow)$: Consider the contrapositive of the converse of $(\rightarrow)$. Suppose $\lim_{x \to p} f(x) \neq q$, then there exists an $\epsilon > 0$ such that for all $\delta > 0$, there exists an $x \in E$ for which $d_y(f(x), q) \geq \epsilon$ but $0 < d_x(x,p) < \delta$. Let $\delta_n = \frac{1}{n} \hspace{0.25cm} (n=1,2,3, \dots)$ and choose $p_n$ such that $0 < d(p_n, p) < \frac{1}{n} = \delta_n$. However, $d(f(p_n), q) \geq \epsilon$. Then, $p_n \to p$ but $f(p_n) \nrightarrow q$, since all images of $p_n$ are at least $\epsilon$ away from $q$. 
\end{proof}

Corollary: if $f$ has a limit at $p$, then the limit is unique. 

\begin{theorem}[\textbf{Theorem 4.4}]
Suppose $(X,d)$ is a metric space, $E \subset X$, $p$ is a limit point of $E$, and both $f$ and $g$ send $E \to \mathbb{R}$. Additionally, $\lim_{x \to p} f(x) = A$ and $\lim_{x \to p} g(x) = B$. Then, 
\begin{enumerate}
    \item $\lim_{x \to p} (f+g)(x) = A + B$
    \item $\lim_{x \to p} (fg)(x) = AB$
    \item $\lim_{x \to p} \left(\frac{f}{g}\right)(x) = \frac{A}{B}$ if $B \neq 0$
\end{enumerate}
\end{theorem}
\begin{proof}
    (Comes from Theorem 3.3 and Theorem 4.2) 
\end{proof}

\section*{Practice Problems}

\begin{enumerate}
    \item Suppose $\Sigma a_n$ converges and $a_n \geq 0$ for all $n$. Show that $\Sigma a_n^2$, $\Sigma \sqrt{a_{n+1} \cdot a_n}$, and $\Sigma \frac{\sqrt{a_n}}{n}$ all converge. (Hint: $(a-b)^2 \geq 0$ for all $a, b \in \mathbb{R}$) 
    \item Suppose we have the following function: 
    $$
    f(x) = \begin{cases} 
            0 & x \in \mathbb{Q} \\
            1 & x \in \mathbb{R} \setminus \mathbb{Q}
        \end{cases}
    $$
    Let $a \in \mathbb{R}$. Show that $\lim_{x \to a} f(x)$ does not exist. 
\end{enumerate}


\end{document}