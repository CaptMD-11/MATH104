\documentclass[12pt]{article}
\usepackage{amsmath}
\usepackage{amsthm}
\usepackage{amsfonts}
\usepackage{amssymb}
\usepackage{authblk}
\usepackage{tkz-euclide}
\usepackage{tikz}
\usepackage{changepage}
\usepackage{lipsum}
\usepackage{tree-dvips}
\usepackage{qtree}
\usepackage[linguistics]{forest}
\usepackage[hidelinks]{hyperref}
\usepackage{mathtools}
\usepackage{blindtext}
\usepackage[cal=esstix,frak=euler,scr=boondox,bb= pazo]{mathalfa}
\usepackage{graphicx}
\graphicspath{{./images/}}
\allowdisplaybreaks
\allowbreak
\theoremstyle{definition}
\newtheorem{definition}{Definition}
\newtheoremstyle{named}{}{}{\itshape}{}{\bfseries}{.}{.5em}{\thmnote{#3's }#1}
\theoremstyle{named}
\newtheorem*{namedconjecture}{Distinct Factorizations Conjecture}
\newtheorem{conjecture}{Conjecture}
\DeclareMathOperator{\sech}{sech}
\DeclareMathOperator{\arcsec}{arcsec}
\newcounter{customDef}
\renewcommand{\thecustomDef}{\arabic{customDef}}
\begin{document}
\title{Math 104 - Tail End Definitions}
\author{}
\date{}
\maketitle
\date

\renewcommand{\thedefinition}{6.1}
\begin{definition}
    
\end{definition}


\renewcommand{\thedefinition}{6.1}
\begin{definition}
A sequence $\{p_n\}$ in a metric space $(X,d)$ is said to be a Cauchy sequence if for every $\epsilon > 0$ there exists an $N \in \mathbb{N}$ such that if $n,m \geq N$, then $d(p_n,p_m) < \epsilon$. 
\end{definition}


\renewcommand{\thedefinition}{6.1}
\begin{definition}
    A metric space in which every Cauchy sequence converges is called complete.
\end{definition}


\renewcommand{\thedefinition}{6.1}
\begin{definition}
    A sequence $\{p_n\}$ of real numbers is said to 
    \begin{enumerate}
        \item be monotonically increasing if $p_n \leq p_{n+1}$ for all $n$.
        \item be monotonically decreasing if $p_n \geq p_{n+1}$ for all $n$. 
    \end{enumerate}
\end{definition}


\renewcommand{\thedefinition}{6.1}
\begin{definition}
    Let $\{s_n\}$ be a sequence of real numbers. Let $E$ be the set of all subsequential limits of $\{s_n\}$. $E$ possibly includes $\infty, -\infty$. Then, let $s_\star = \inf E$ to be the lower limit of $\{s_n\}$ and $s^\star = \sup E$ be the upper limit of $\{s_n\}$. Then, it follows that $s^\star = \lim sup s_n$ and $s_\star = \lim \inf s_n$. 
\end{definition}


\renewcommand{\thedefinition}{6.1}
\begin{definition}
    Given a sequence $\{a_i\}$, let $\sum_{i=p}^{q} a_i = a_p + a_{p+1} + \dots + a_q$. Then $s_n = \sum_{i=1}^{n} a_i$ is the $n^{th}$ partial sum. 
\end{definition}


\renewcommand{\thedefinition}{6.1}
\begin{definition}
    Let $(X,d_x)$ and $(Y,d_y)$ be metric spaces, $E \subset X$, $f: E \to Y$, and $p \in E'$. Then, $\lim_{x \to p} f(x) = q$ if for every $\epsilon > 0$ there exists a $\delta > 0$ such that for all $x \in E$, if $0 < d_x(x,p) < \delta$, then $d_y(f(x), q) < \epsilon$. 
\end{definition}


\renewcommand{\thedefinition}{6.1}
\begin{definition}
    Use the same preliminaries as before except $p \in E$. Then $f$ is continuous at $p$ if for every $\epsilon > 0$ there exists a $\delta > 0$ such that for all $x \in E$, if $d_x(x,p) < \delta$ then $d_y(f(x), f(p)) < \epsilon$. 
\end{definition}


\renewcommand{\thedefinition}{6.1}
\begin{definition}
    A mapping $f: X \to \mathbb{R}^k$ is bounded if there exists an $M \mathbb{R}$ such that $|f(x)| \leq M$ for all $x \in E$. 
\end{definition}


\renewcommand{\thedefinition}{6.1}
\begin{definition}
    Let $f: X \to Y$. $f$ is uniformly continuous on $X$ if for every $\epsilon > 0$ there exists a $\delta > 0$ such that for all $p,q \in X$ if $d_x(p,q) < \delta$ then $d_y(f(p), f(q)) < \epsilon$. 
\end{definition}


\renewcommand{\thedefinition}{6.1}
\begin{definition}
    Let $f: [a,b] \to \mathbb{R}$. $f$ is differentiable at $x \in [a,b]$ if the following limit exists: 
    $$
    f'(x) = \lim_{t \to x} \frac{f(t) - f(x)}{t-x}
    $$
    where $f'$ is the first derivative of $f$ and $t \in [a,b]$ with $t \neq x$. 
\end{definition}

\renewcommand{\thedefinition}{6.1}
\begin{definition}
    Let $X$ be a set. $X$ is sequentially compact if every sequence in $X$ has a subsequence that converges to a point in $X$. 
\end{definition}

\renewcommand{\thedefinition}{6.1}
\begin{definition}
    Let $f: X \to \mathbb{R}$. $f$ has a local minimum at $p$ if there exists a $\delta > 0$ such that $f(q) \geq f(p)$ for every $q \in X$ with $d(p,q) < \delta$. $f$ has a local maximum at $p$ if there exists a $\delta > 0$ such that $f(q) \leq f(p)$ for every $q \in X$ with $d(p,q) < \delta$. 
\end{definition}

\renewcommand{\thedefinition}{6.1}
\begin{definition}
    Let $[a,b]$ be the given interval. A partition $P$ is a set of points $x_0, \dots, x_n$ with $a=x_0 \leq x_1 \leq \dots \leq x_n = b$. Then, $\Delta x_i = x_i - x_{i-1}$ and let $f$ be a bounded function on $[a,b]$. Then, let $M_i := \sup f(x)$ on $[x_{i-1}, x_i]$ and $m_i := \inf f(x)$ on $[x_{i-1}, x_i]$. Then, let $L(P,f) = \sum_{i=1}^{n} m_i \Delta x_i$ and $U(P,f) = \sum_{i=1}^{n} M_i \Delta x_i$. Then, let $\int_{\underline{a}}^{b} f dx = \sup L(P,f)$ be the lower Riemann integral of $f$ on $[a,b]$ and let $\int_{a}^{\overline{b}} f dx = \inf U(P,f)$ be the upper Riemann integral of $f$ on $[a,b]$. Then, if $\int_{\underline{a}}^{b} f dx = \int_{a}^{\overline{b}} f dx$, then $f$ is Riemann integrable on $[a,b]$ and denote the common value as $\int_{a}^{b} f dx$ (called the Riemann integral of $f$ on $[a,b]$). 
\end{definition}

\renewcommand{\thedefinition}{6.1}
\begin{definition}
    Let $\alpha$ be a monotonically increasing function with $\Delta \alpha_i = \alpha(x_i) - \alpha(x_{i-1})$. Note that $\Delta \alpha_i \geq 0$ since $\alpha$ is monotonically increasing. Let $f$ be a real bounded function on $[a,b]$. Then, let $L(P,f,\alpha) = \sum_{i=1}^{n} m_i \Delta \alpha_i$ and $U(P,f,\alpha) = \sum_{i=1}^{n} M_i \Delta \alpha_i$. Then, let $\int_{\underline{a}}^{b} f d\alpha = \sup L(P,f,\alpha)$ and $\int_{a}^{\overline{b}} f d\alpha = \inf U(P,f,\alpha)$. If $\int_{\underline{a}}^{b} f d\alpha = \int_{a}^{\overline{b}} f d\alpha$ then $f$ is Riemann-Stieltjes integrable on $[a,b]$ and denote the common value as $\int_{a}^{b} f d\alpha$. 
\end{definition}

\renewcommand{\thedefinition}{6.1}
\begin{definition}
    The partition $P'$ is a refinement of the partition $P$ if $P' \supset P$. $P$ is the common refinement of partitions $P_1$ and $P_2$ if $P = P_1 \cup P_2$. 
\end{definition}

\renewcommand{\thedefinition}{6.1}
\begin{definition}
    Suppose $\{f_n\}$ is a sequence of functions defined on $E$ and $\{f_n(x)\}$ is a sequence of numbers that converges for every $x \in E$. Then, define the following function: 
    $$
    f(x) = \lim_{n \to \infty} f_n(x)
    $$ 
    for all $x \in E$. Then, $\{f_n\}$ converges on $E$ if $f$ is the limit of $\{f_n\}$. $\{f_n\}$ converges pointwise to $f$ on $E$. If $\sum f_n(x)$ converges for every $x \in E$, and if we define
    $$
    f(x) = \sum_{n=1}^{\infty} f_n(x)
    $$
    then $f$ is the sum of the series $\sum f_n$. 
\end{definition}

\renewcommand{\thedefinition}{6.1}
\begin{definition}
    Let $f$ be a bounded function with $f: E \to \mathbb{R}$. Then, let the following represent the norm of $f$. 
    $$
    ||f|| = \sup_{x \in E} |f(x)|
    $$
\end{definition}

\renewcommand{\thedefinition}{6.1}
\begin{definition}
    $\{f_n\}$ converges uniformly if for every $\epsilon > 0$ there exists an $N \in \mathbb{N}$ such that if $n \geq N$, then, $||f_n - f|| < \epsilon$ for every $x \in E$. 
\end{definition}

\renewcommand{\thedefinition}{6.1}
\begin{definition}
    If $(X,d)$, let $\mathscr{C}(x) = \{f: X \to \mathbb{C}: f \text{ is bounded and continous}\}$. Then, for $f \in \mathscr{C}(x)$, let its supremum norm be defined by $||f|| = \sup_{x \in X} |f(x)|$. Also define $d_{\mathscr{C}(x)} (f,g) = ||f-g||$, where $\mathscr{C}(x)$ is a metric space. 
\end{definition}

\renewcommand{\thedefinition}{6.1}
\begin{definition}
    Let $\{f_n\}$ be a sequence of bounded functions on $E$. $\{f_n\}$ is pointwise bounded if $\{f_n(x)\}$ is bounded for every $x \in E$, that is, there is a real-valued function $\phi$, also defined on $E$, such that $|f_n(x)| < \phi$ for every $x \in E$. $\{f_n\}$ is uniformly bounded on $E$ if there exists an $M \in \mathbb{R}$ such that $|f_n(x)| \leq M$ for every $x \in E$ and natural number $n$.  
\end{definition}

\end{document}